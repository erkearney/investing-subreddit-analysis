\documentclass[11pt]{article}

\usepackage{hyperref}
\usepackage{graphicx}
%\usepackage{enumitem}
%\usepackage{setspace}
%\usepackage{geometry}

\title{Simulating the portfolios of investing subreddits using sentiment analysis}
\author{CS-542 final project}
\author{Eric Kearney}
\date{\today}

\begin{document}
\maketitle
\thispagestyle{empty}

\section{Introduction}
    \href{reddit.com}{reddit.com} is a popular online discussion board, divided
    into "subreddits", which are user-created and dedicated to a particular
    topic of discussion. This means that there are often multiple subreddits
    dedicated to the same topic broadly, however because they grew separately
    they have often fostered their own individual communities, each with their
    own beliefs, rules, goals, etc.\\

    I've identified three subreddits focused on investing into the U.S. stock
    market, \href{https://www.reddit.com/r/investing/}{r/investing},
    \href{https://www.reddit.com/r/stocks/}{r/stocks}, 
    and \href{https://www.reddit.com/r/wallstreetbets/}{r/wallstreetbets}
    (content warning: These online communities, in particular r/wallstreetbets,
    are known for using language that some may find offensive).\\

    The goal of this project was to use sentiment analysis to determine how
    each of these subreddits viewed individual stocks, assemble a portfolio
    for each subreddit based on this analysis, and compare the performance of
    these portfolios against each other.\\
\section{Collecting data}
    I started this project with this
    \href{https://www.kaggle.com/datasets/yorkehead/stock-market-subreddits}
    {Stock Market Subreddits dataset} I found on
    \href{https://www.kaggle.com/}{kaggle}. I then wrote a script
    (collect\_stock\_data) using
    \href{https://pypi.org/project/yahoo-finance/}{this API} which uses Yahoo
    Finance to allow me to download historic Stock Market Data.
\section{Data Pre-Processing}
    The next task for this project was the pre-process of 'clean' the reddit
    dataset. I began with boilerplate language processing techniques, such as
    using \href{https://www.nltk.org/}{nltk} to remove words that are common,
    yet carry little meaning (ex: "The", "I", "What", etc.). Finally, I
    removed all puncuation from the text.\\

    I then scanned each post and comment in the reddit dataset for text that
    could be found in the list of NASDAQ symbols obtained from
    \href{https://www.nasdaqtrader.com/dynamic/SymDir/nasdaqtraded.txt}
    {nasdaqtrader.com} (e.g., 'AAPL' for 'Apple', 'TSLA' for 'Tesla'), once a
    symbol was found, I attached it as a new column in the dataframe to that
    post. If a single post contained multiple symbols, I copied the post and
    added a new entry into the dataset, with the second symbol attached in the
    new column, and repeated as necessary (e.g., a post titled: "Video
    explaining the huge growth of large cap growth stocks (AAPL, AMZN, MSFT,
    TSLA, FB, GOOGL)","I've seen this question everywhere on this sub: ""why
    did apple/tesla/[insert large growth company name here] growth so much so
    quick! They still haven't made a lot of profit! We are in a bubble!"* from
    r/investing became:\\

    \noindent'AAPL', 'Video explaining the huge growth of large cap . . .'\\
    'AMZN', 'Video explaining the huge growth of large cap . . .'\\
    'MFTS', 'Video explaining the huge growth of large cap . . .'\\
    etc.)\\

    *Unfortunately the Kaggle dataset did not include the usernames of those
    who posted, so I cannot properly give credit to this quote.
\section{Using VADER}
   For the sentiment analysis portion of this project, I elected to use
   \href{https://github.com/cjhutto/vaderSentiment}{VADER} (Valence Aware
   Dictionary and sEntiment Reasoner)[1]. VADER is a pre-trained sentiment
   analysis model trained using social media data, making it a powerful tool
   for this task. One potential cavaet is that the subreddit r/wallstreetbets
   is well-known for having a more unique lexicon and culture, which could
   throw a wrench in VADER's ability to determine the sentiment of statements
   originating from it.\\

   I had originally hoped to be able to train my own model, which would have
   naturally led to more accurate results, however due to a shortage of time
   and resources I decided it would be more prudent to simply use a
   pre-trained model instead.\\

   \begin{small}
   \href{https://ojs.aaai.org/index.php/ICWSM/article/view/14550/14399}
   {[1]VADER: A Parsimonious Rule-based Model for Sentiment Analysis of Social Media Text
   (by C.J. Hutto and Eric Gilbert)
   Eighth International Conference on Weblogs and Social Media (ICWSM-14). Ann Arbor, MI, June 2014.}
   \end{small}
\section{Simulating the Stock Market}
    Using the analyzed posts and comments from the three subreddits along with
    the historic stock market data, I created a stock market simulation, in
    which three traders, one for each subreddit, exists. My simulation began on
    April 27th, 2019 and ended January 27th, 2021. On each day between those
    dates, the simulation found all posts/comments from the reddit dataset
    that were posted on that day, then if the sentiment expressed in that
    post was positive, the trader representing the subreddit from which the
    post originated from would 'buy' one share of the stock associated with
    that post. If, on the other hand, the post had a negative sentiment, the
    trader would 'sell' one share of that stock (assuming it actually owned a
    share of it to begin with). There were times where there'd be multiple
    posts concerning the same stock from the same subreddit, and the sentiment
    in each of those posts diverged. To handle this situation, I created a
    'score' which was caluclated by using the number of 'upvotes' each post
    had received from other members in the community. I multiplied the upvotes
    of posts with a negative sentiment by -1, and then summed the score for
    all the posts concerning the same stock together. If that score ended up
    being > 0, the trader 'bought' the stock, otherwise, it 'sold' the stock.\\

    My intention was to simulate around five years of the stock market, however
    I realized that r/wallstreetbets hasn't been active that long, so instead I
    had to scale back to simulating 21 months.
\section{Results}
    The following is the value of all the stocks in the portfolio of each
    subreddit, minus the amount that it would've taken to buy those stocks,
    on the last day of the simulation.\\
    \begin{figure}
	\centering
	\includegraphics[width=\textwidth]{data/results/results.png}
    \end{figure}
    \begin{itemize}
	\item r/stocks : \$147,138.64
	\item r/investing : \$32,556.34
	\item r/wallstreetbets : \$7,302.13
    \end{itemize}

    While it appears that investing was the clear winner, there is much more to
    this story (See Errors \& Future Work)
\section{Errors \& Future Work}
    While developing this project, it became clear that there are many errors
    currently embedded within it. Resolving these errors will require a great
    deal of time and effort, and I do have to turn in \textbf{something}. I
    hope to continue work on this as a personal project after class ends. Here
    are the errors I am currently aware of:
    \begin{itemize}
	\item The pre-processing currently only looks for NASDAQ symbols
	    within the text, not for company names. For example, a post that
	    contains 'AAPL' will be correctly flagged as discussing Apple, Inc.,
	    however a post containing 'Apple' will not. This means that not all
	    the companies being discussed are being correctly 'bought' in the
	    simulation, and that companies with names that are commonly
	    abbreviated to their NASDAQ symbols, such as 'GME', will be
	    overrepresented in the simulation compared to companies that are
	    often fully spelled out when discussed online.
	\item Some companies have NASDAQ symbols that happen to be equivalent
	    to common words, most apparently is Agilent Technologies Inc.,
	    which has the NASDAQ symbol 'A'. This means that stocks were being
	    'bought' when they shouldn't have been. The subreddit hurt most by
	    this error was clearly r/wallstreet bets. Whose portfolio was
	    comprised almost entirely of 'YOLO', which is the AdvisorShares
	    Pure Cannabis ETF. It is clear from the context of the comments
	    that the intention behind saying 'YOLO' was almost always
	    as the acronym "You Only Live Once". An example comment containing
	    "YOLO" from r/wallstreetbets:\\

	    "Keynes said we're all dead in the long run, I think he meant
	    always YOLO"*
	\item Another issue that particularly hurt r/wallstreetbets were
	    sentiment mis-identifications. For example the post:\\

	    "CLASS ACTION AGAINST ROBINHOOD. Allowing people to only sell is
	    the definition of market manipulation. A class action must be
	    started, Robinhood has made plenty of money off selling info about
	    our trades to the hedge funds to be able to pay out a little for
	    causing people to loose money now","LEAVE ROBINHOOD. They dont
	    deserve to make money off us after the millions they caused in
	    losses. It might take a couple of days, but send Robinhood to the
	    ground and GME to the moon."*\\

	    Scored a negative sentiment for GME. While this appears to be
	    correct as in the sentiment of the statement itself is negative,
	    that negativity is not intended to be directed towards GME. As a
	    matter of fact, a great deal of the discussion on r/wallstreetbets
	    was targeted towards this Robinhood class action lawsuit, and
	    similar lawsuits. The end result of this was that r/wallstreetbets
	    bought almost no shares of GME, which is the community's stock of
	    choice.
	\item Subreddits were allowed to buy or sell no more than one share of
	    each company a day. A better simulation probably would allow for
	    the purchase of numerous shares of a stock if the subreddit's
	    sentiment towards that company was extremely positive, and
	    conversely, allow for the sale of numerous shares if the sentiment
	    were extremely negative.
	\item Clearly, more data from each subreddit, as well as model trained
	    on the text of the subreddits themselves would've improved
	    accuracy.
	\end{itemize}

	*I unfortunately cannot quote the users as their
	usernames were not included in the dataset.
\section{Limitations}
    It should be made explicity clear to anyone who wishes to work on a
    project like this, make a similar project, or base their financial
    decisions based on a project like this that it is incredbily easy to
    manipulate the results. Almost any portfolio can be made to look
    incredible if one is willing to cherry pick the dates. I personally would
    under no circumstances recommend that anyone invest money based on the
    advice of strangers on the Internet, without thoroughly doing their own
    research. It may be tempting to start following r/stocks now and just
    invest in whatever the users there seem to be buying, but it should
    always be stated again and again that past performance does not indicative
    of future results.
\end{document}
